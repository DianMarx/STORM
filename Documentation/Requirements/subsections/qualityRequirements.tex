The quality requirements are the requirements around the quality attributes of the system and
the services it provides. Quality requirements relevant to project STORM are listed below in order of priority. \par

\subsubsection{Usability} \label{3.2.1}
Usability is one of the most important quality attributes. Usability tests should be performed to check whether:
\begin{itemize}
\item users find any aspects of the system cumbersome, non-intuitive or frustrating.
\item the user has a positive experience and finds the functionality easy to use and learn.
\end{itemize}

\subsubsection{Scalability} \label{3.2.2}
The system must implement a very generic optimization algorithm in order to be used by different parties in different contexts. In addition to different environments the system should be able to optimize groups for an extremely large number of subjects.

\subsubsection{Performance} \label{3.2.3}
\begin{itemize}
\item The optimization algorithm should be able to sort a maximum of 10 000 subjects into groups and respond within 10 seconds.
\item Reporting queries should respond within 15 seconds. \par
\textit {*The above figures does not include the network round-trip which is outside the control of the system.}
\end{itemize}

\subsubsection{Reliability} \label{3.2.4}
The system should provide by default a reasonable level of reliability and should be deployable within configurations which provide a high level of availability, supporting
\begin{itemize}
\item fail-over safety of all components and
\item a deployment without single points of failure.
\end{itemize}
Hot deployment of new or changing functionality is not required for this system.

\subsubsection{Deployability} \label{3.2.5}
The system must be deployable
\begin{itemize}
\item on Linux servers,
\item and in environments using different databases for persistence of the STORM data.
\end{itemize}

\subsubsection{Security} \label{3.2.6}
Security is a fairly important aspect of the system in order to protect sensitive project and subject data.
The system should also allow users that are logged on to the system, to assign collaborators
to a project. 


