\begin{flushleft}
The system is to be a modular system which allows for:
\end{flushleft}

\begin{itemize} 
\item[$\bullet$] Only a subset of modules to be deployed minimally the system will require the core modules to be deployed.
\item[$\bullet$] Further modules to be added at a later stage.
\end{itemize}

\begin{flushleft}
To this end there should be:
\end{flushleft}

\begin{itemize} 
\item[$\bullet$] Minimal dependencies between modules, and
\item[$\bullet$] No dependencies of core modules on any add-on modules.
\end{itemize}

\begin{flushleft}
Modular design allows that each module encapsulates information that is not available to the rest of a program. This information hiding reduces the cost of subsequent design changes when future functionality is added to STORM. For example, if at a later stage functionality is added to allow for personality tests to be completed within STORM and then results are automatically pulled in, a new module can be added without affecting other modules.
\end{flushleft}