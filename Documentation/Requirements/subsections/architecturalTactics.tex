This section discusses the architectural tactics which are used to concretely address the quality
requirements for the application

\subsubsection{Database abstraction}
The system includes a database abstraction module to improve:
\begin {itemize}
\item Deplayability as mentioned in section \ref {3.2.5} in environments where different databases are used,
\item Security as mentioned in section \ref {3.2.6} to isolate database and hide sesitive subject data.
\end {itemize}

\subsubsection{UI components framework}
The system will use a rich, dynamic JQuery-UI component library in order to:
\begin {itemize}
\item provide a rich, dynamic user interface for usability (requirement \ref {3.2.1}),
\item improve scalability (requirement \ref {3.2.2}) and performance (requirement \ref {3.2.3}).
\end{itemize}

\subsubsection{Maximize client-side scripting}
The system will use client-side javascript where possible to address:
\begin {itemize}
\item Performance (requirement \ref {3.2.3}) by avoiding the netork round trip,
\item scalability (requirement \ref {3.2.2}) and security (requirement \ref {3.2.6}) by hiding data out of a user's bounds.
\end{itemize}

\subsubsection{MVC}
The system will use a MVC pattern to address:
\begin {itemize}
\item Usability (requirement \ref {3.2.1}),
\item Deployability (requirement \ref {3.2.5}).
\end{itemize}